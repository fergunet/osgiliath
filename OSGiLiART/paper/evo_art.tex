% !TEX encoding = UTF-8 Unicode
\documentclass{llncs}
\usepackage{llncsdoc}
\usepackage[T1]{fontenc}
\usepackage[utf8]{inputenc}
\usepackage{geometry}                % See geometry.pdf to learn the layout options. There are lots.
\geometry{letterpaper}                   % ... or a4paper or a5paper or ... 
%\geometry{landscape}                % Activate for for rotated page geometry
%\usepackage[parfill]{parskip}    % Activate to begin paragraphs with an empty line rather than an indent
\usepackage{graphicx}
\usepackage{amssymb}
\usepackage{epstopdf}
\DeclareGraphicsRule{.tif}{png}{.png}{`convert #1 `dirname #1`/`basename #1 .tif`.png}

\title{Evolutionary Art}
\author{}
%\date{}                                           % Activate to display a given date or no date

\begin{document}
\maketitle

\begin{abstract}
<Text of the summary of your article>
\end{abstract}

\section{Introudction}

\section{State of the art}

\subsection{Aesthetic measures for evolutive art}
According to Galanter \cite{galanter2012computational}, computational aesthetics measures can be classified in the following categories:
\begin{itemize}
	\item Based on Formulaic and Geometric Theories.
	\item Based in Design Principles.
	\item Based in Neural Networks and Connective Models.
	\item Based in Evolutionary Systems:
		\begin{itemize}
			\item Interactive Evolutionary Computation.
			\item Performance based goals.
			\item Error relative to Exemplars.
			\item Complexity measures.
			\item Multi-objective.
			\item Extensions to EA (such as, coevolution, agent swarm behavior, etc.).
		\end{itemize}
	\item Complexity Based Models
\end{itemize}

%\subsection{}

\bibliographystyle{plain}

\bibliography{evo_art}

\end{document}  